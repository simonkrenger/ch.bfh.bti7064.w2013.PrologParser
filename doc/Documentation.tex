\documentclass{article}
\usepackage{amsmath}
\usepackage{amsthm}
\usepackage{amssymb}
\usepackage[utf8]{inputenc} 
\usepackage{tikz}
\usepackage{listings}
\usepackage{courier}
\usetikzlibrary{automata,positioning}

\title{Prolog Parser}
\author{Franziska Corradi \{corrf1@bfh.ch\}\\Simon Krenger \{krens1@bfh.ch\}}

\begin{document}
\maketitle
\section{Einleitung}
Das vorliegende Projekt verwendet einen deterministischen endlichen Automaten, um die Tokens des Prolog-Programms zu analysieren.

\section{Übergangstabelle}
In der Datei \lq\lq{}doc/Transitionstabelle.xlsx\rq\rq{} findet sich die vom Programm verwendete Transitionstabelle.

\section{Programmbeschrieb}
Das Kommandozeilenprogramm (JAR) soll mittels des folgenden Befehl gestartet werden:
\begin{lstlisting}
$ java -jar PrologParser.jar inputfile.pl
\end{lstlisting}
Das Programm selbst verwendet die folgenden Klassen:
\begin{itemize}
\item \underline{PrologParser}: Einstiegspunkt des Programms. Diese Klasse enthält die \texttt{main()} Methode und erstellt die Instanzen der anderen Klassen.
\item \underline{PrologPreprocessor}: Diese Klasse wird dazu verwendet, den Quellcode einzulesen und für den Scanner vorzubereiten. Im Preprocessor werden alle Kommentare entfernt, alle Spezialzeichen entfernt (Methode \texttt{preprocess()}).
\item \underline{PrologScanner}: Diese Klasse stellt den Hauptteil des Programms dar. Hier wird der Automat erstellt und die Transitionen definiert. Die \texttt{tokenize(String input)} Methode liefert dann die Tokens zurück.
\item \underline{LexerState}: Klasse für den Automaten, repräsentiert einen Zustand des Automaten.
\item \underline{LexerStateTranslation}: Klasse für den Automaten, repräsentiert eine Zustandsänderung des Automaten.

\end{itemize}

\newpage
\section{Deterministischer endlicher Automat}
Folgender Graph zeigt den Automaten für den vereinfachten PrologParser.\\\\
\begin{tikzpicture}[shorten >=1pt,node distance=3cm,auto] 
	\node[state,initial] (q_0) {$INIT$}; 
	\node[state] (q_1) [right=of q_0] {$CONST$}; 
	\node[state] (q_2) [below=of q_1] {$SPECIAL$}; 
	\node[state] (q_3) [left =of q_2] {$VARIABLE$}; 
	\node[state] (q_4) [below=of q_2] {$COLON$}; 
	\node[state] (q_5) [below=of q_4] {$DASH$}; 
	\path[->] 
		(q_0) edge node {[a-z]} (q_1)
		(q_0) edge [bend right=10] node [swap] {[A-Z0-9\_]} (q_3)
		(q_0) edge [bend right=10] node [swap] {[\textbackslash W]} (q_2)
		
		(q_1) edge [bend left=10] node {[\textbackslash W]} (q_2)
		edge [loop above] node {[a-z0-9]} ()

		(q_2) edge [bend left=10]  node {[a-z0-9]} (q_1)
		(q_2) edge [bend left=10]  node {[A-Z\_]} (q_3)
		(q_2) edge node {\lq{}:\rq{}} (q_4)
		edge [loop right] node {[\textbackslash W]} ()

		(q_3) edge [bend left=10] node {[\textbackslash W]} (q_2)
		edge [loop left] node {[A-Za-z0-9\_]} ()

		(q_4) edge node {\lq{}-\rq{}} (q_5)

		(q_5) edge [bend left=10] node {[A-Z\_]} (q_3)
		(q_5) edge [bend left=20] node {[\textbackslash W]} (q_2)
		(q_5) edge [bend right=60] node {[a-z0-9]} (q_1)
		;
\end{tikzpicture}
\end{document}  